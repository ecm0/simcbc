%%%%%%%%%%%%%%%%%%%%%%%%%%%%%%%%%%%%%%%%%
% a0poster Portrait Poster
% LaTeX Template
% Version 1.0 (22/06/13)
%
% The a0poster class was created by:
% Gerlinde Kettl and Matthias Weiser (tex@kettl.de)
% 
% This template has been downloaded from:
% http://www.LaTeXTemplates.com
%
% License:
% CC BY-NC-SA 3.0 (http://creativecommons.org/licenses/by-nc-sa/3.0/)
%
%%%%%%%%%%%%%%%%%%%%%%%%%%%%%%%%%%%%%%%%%

%----------------------------------------------------------------------------------------
%	PACKAGES AND OTHER DOCUMENT CONFIGURATIONS
%----------------------------------------------------------------------------------------

\documentclass[a0,portrait]{a0poster}

\usepackage{subcaption} 

\usepackage{ wasysym } % astrosun symbol.
\usepackage{gensymb} % degree symbol
\usepackage{multicol} % This is so we can have multiple columns of text side-by-side
\columnsep=100pt % This is the amount of white space between the columns in the poster
\columnseprule=3pt % This is the thickness of the black line between the columns in the poster

\usepackage[svgnames]{xcolor} % Specify colors by their 'svgnames', for a full list of all colors available see here: http://www.latextemplates.com/svgnames-colors

%\usepackage{times} % Use the times font
\usepackage{palatino} % Uncomment to use the Palatino font

\usepackage{graphicx} % Required for including images
\graphicspath{{figures/}} % Location of the graphics files
%\usepackage{booktabs} % Top and bottom rules for table
\usepackage[font=small,labelfont=bf]{caption} % Required for specifying captions to tables and figures
%\usepackage{amsfonts, amsmath, amsthm, amssymb} % For math fonts, symbols and environments
\usepackage{wrapfig} % Allows wrapping text around tables and figures
\usepackage{url}

%\usepackage{realboxes}

\renewcommand{\emph}[1]{\textbf{\color{blue}#1}}

\newcommand{\vs}[1] {\textbf{\textcolor{red}{#1}}}

\newcommand{\ECM}[1] {\textbf{\textcolor{magenta}{#1}}}

%\renewcommand{\section}[2]{\Colorbox{lightgray}{\noindent {\Large \textbf{#2}} \hfill}}

%\usepackage{lipsum}

\begin{document}

%----------------------------------------------------------------------------------------
%	POSTER HEADER 
%----------------------------------------------------------------------------------------

% The header is divided into two boxes:
% The first is 75% wide and houses the title, subtitle, names, university/organization and contact information
% The second is 25% wide and houses a logo for your university/organization or a photo of you
% The widths of these boxes can be easily edited to accommodate your content as you see fit

\begin{minipage}[b]{0.75\linewidth}
  \Huge \textbf{ 	Prospects about X- and gamma-ray counterparts of gravitational wave signals with INTEGRAL}\\[1cm] % Title
  \large \textbf{P.~Bacon, V.~Savchenko and E.~Chassande-Mottin}\\[1cm] % Author(s)
  \normalsize APC, Univ Paris Diderot, CNRS/IN2P3, CEA/Irfu, Obs. de Paris, Sorbonne Paris Cit\'e, France\\
  \large \texttt{philippe.bacon@apc.in2p3.fr}\\
\end{minipage}
%
\begin{minipage}[b]{0.25\linewidth}
	\includegraphics[scale=.8]{logo.png}
\end{minipage}

\vspace{1cm} % A bit of extra whitespace between the header and poster content

%----------------------------------------------------------------------------------------

\begin{multicols}{2} % This is how many columns your poster will be broken into, a portrait poster is generally split into 2 columns


 \begin{abstract}
   By extrapolating the number of detections made during the first LIGO science
   run, tenths of gravitational wave signals from binary black hole mergers are
   anticipated in upcoming LIGO Virgo science runs. Finding an electromagnetic
   counterpart to compact binary merger events would be a landmark
   discovery. The search for such counterpart is challenging for a number of
   reasons, such as the poor resolution of source position reconstruction from
   the gravitational wave observations alone, and the weakness of the expected
   electromagnetic signal. In this poster, we evaluate the ability of current
   wide-field X- and gamma-ray telescopes aboard INTEGRAL to find such
   counterparts. We present the result of an end-to-end simulation for
   estimating the fraction of the sources that can be followed up, and the
   fraction of counterparts that can be detected based on different models.
 \end{abstract}

 % satellite : What fraction of GW events can be recovered in the EM spectrum ?
 % What should be the statistical significance of the GW and the EM events to
 % claim a confident joint detection ? \textcolor{blue}{BOTTOM LINE RESULTS : We
 % show that...}

\section*{Motivations}

%\ECM{Intro on LIGO Virgo -- Objective of the study and major steps.}
On September 2015, the two LIGO interferometers realized the first direct detection of gravitational waves (GW) and inaugurated a multi-messenger era in astronomy history. A next step would be to detect some electromagnetic (EM) counterparts associated to GW events during the second observing run (O2). Among them short gamma-ray bursts (SGRBs) produced by the coalescence of binary neutron stars are surely the most promising sources detectable by high energy detectors. We thus address the question of the joint detectability of GW events by Advanced Virgo and Advanced LIGO interferometers, and EM events with the INTEGRAL mission. \\
\emph{We produce a end-to-end simulation of the search for gravitational-waves
from binary neutron star mergers \textit{and} follow-up observation seeking a
 possible electromagnetic counterpart at high-energies.} This study is similar
to the one done in \cite{2016arXiv160606124P} with Fermi.

\section*{Monte-Carlo simulation of gravitational-wave events from neutron-star binary mergers}

We start with a simulated catalogue of 6 millions Milky Way-like galaxies
distributed to $z\sim 0.12$. To those galaxies, we associate binary neutron star
(BNS) mergers with mass distribution and rate of R=$23.5 M\mathrm{yr}^-3$ per galaxy
according to Model A ($Z_{\astrosun}$ metallicity) of
\cite{dominik12:_doubl_compac_objec}. This results in a population of 14\,000
mergers for a $100$ year simulation.

\begin{center}\vspace{.5cm}
    \includegraphics[width=30cm]{figures/summary_plot.png}
\end{center}

We simulate GW signals corresponding to the binary mergers (using model
``TaylorT4'') and inject them into Gaussian noise coloured according to the
``optimistic'' power-spectral density anticipated for the 2nd LIGO/Virgo run O2
\cite{lrr-2016-1}. We select signals with $\mathrm{SNR}_{\mathrm{combined}} > 9$
corresponding to a false-alarm rate $\mathrm{FAR} < 1/\mathrm{month}$. The average range
BNS is 126 Mpc for this selection.

The resulting 232 binaries are distributed randomly in time during the expected
period of O2 from Nov 2016 to May 2017 (6 months). We recover GW signals using
matched-filtering techniques and reconstruct the posterior skymaps giving the
position of the source with the BAYESTAR pipeline \cite{PhysRevD.93.024013}.

% SNR threshold HL = 6 : RHO_thres_NETWORK = sqrt(rho_H**2 + rho_L**2) = 6
% On one hand, the signal only takes the inspiral phase into account via the use of non-spinning waveform 'TaylorF4ThreePN'.
% As we want to reproduce an analogue study with \textsc{INTEGRAL} it justifies why our simulation settings are aligned on their work. This will allow further comparisons.
% As a transition towards the details of the EM emission we estimated the total GW emitted energy $E_{\mathrm{EM}}$ of the simulated BNS systems thanks to Bernuzzi et al. \cite{2016PhRvD..94b4023B}, ie.  $E_{GW} \sim 1.5 \% \, M c^{2}$ where M is the total mass of the binary.

\section*{Electromagnetic emission model from neutron-star binary mergers}

%\ECM{Tell: Connection BNS / Gamma-ray bursts, simplified GRB emission model with basic features}

The coalescence of BNS systems is known to be a potential candidate for SGRBs. As the material from the two neutron stars collide, the tremendous amount of energy implied in the process accelerate the matter. It results in the emission of a non-thermal spectrum in the gamma-ray domain (mostly synchrotron). \\
Observational evidences of GRBs generally show two kinds of EM emissions : a prompt emission in the gamma-ray or X-ray domain which corresponds to the collimated ejection of relativistic jets of matter, and an afterglow emission in the X-ray, optic or radio domain produced after the ejecta has been slowed down by internally driven shocks between many layers. \\
Here we present a simplified GRB emission model with some basic feature. For instance, every simulated GW event is associated to a gamma-ray event and the total emitted EM energy is not dependant on the BNS inclination with respect to the line of sight.

\subsection*{Prompt emission}

%\ECM{Tell: light curve evolution, isotropic emitted energy and beaming}

Luminosity curves associated to the prompt emission show a huge variability of patterns but some features enable use to disentangle short gamma-ray burst ($\sim 100 \, ms$) from long gamma-ray bursts ($\sim 10 \, ms$). However both present an abrupt peak immediately followed by a damped decay. \\
\indent The prompt emission is well modelled by the Band emission spectrum with the following fixed parameters $\alpha_{\mathrm{BAND}} = - 0.5$, $\, \beta_{\mathrm{BAND}} = - 2.25$ and $E_{\mathrm{peak}} = 800 \, \mathrm{MeV}$. As it is associated to the coalescence of the BNS it is not expected to last more than a few seconds. 

\subsection*{Afterglow}

%\ECM{Tell: light curve evolution, isotropic emitted energy and beaming}

After the ejecta has been slowed down, the extremely beamed emission tends to open up as it looses energy (GRB is mostly visible in soft X-ray). The luminosity curve rapidly decreases after prompt emission and can be modelled by a cut-off power law with $\lambda = - 0.5$ and $E_{\mathrm{peak}} = 600 \, \mathrm{keV}$. Contrary to the prompt emission the afterglow lasts from several days to several weeks (in the radio domain) after the coalescence. \\
\indent In our model we considered a follow-up occurring at late afterglow. In the absence of any beaming the emission is taken to be isotropic.


% Power law model see Eq(1) of 'THE FERMI GBM GAMMA-RAY BURST SPECTRAL CATALOG: FOUR YEARS OF DATA' - Gruber et al 

\begin{center}\vspace{.5cm}
    \includegraphics[width=20cm]{figures/spectra.png}\includegraphics[scale=.65]{figures/Integralinstru.jpg}
    \captionof{figure}{(\textit{Left}) Prompt and afterglow emission spectra from SGRBs systems. Fluxes have been normalized with respect to the INTEGRAL finite energy band $\left[ 75 \, , 2000 \right] \, \mathrm{keV}$. (\textit{Right}) Instruments embarked on INTEGRAL satellite.}
    \label{spectra}
\end{center}

\section*{The INTEGRAL mission and possible follow-up strategies}

%\ECM{Describe the relevant instruments here. Energy range covered. FOV.}
% LINK : http://www.cosmos.esa.int/web/integral/instruments-ibis
%        http://www.cosmos.esa.int/web/integral/instruments-spi

The EM follow-up strategy is partially determined by instrumental performances. The INTEGRAL satellite embarks to instruments : the SPI- ACS gamma-ray spectrometer and the IBIS imager. The former takes advantage of a very wide $16^{\circ}$-field of view (FoV) essential to monitor prompt emissions (passive follow-up) in the  $\left[ 18 \, \mathrm{keV}, 8 \, \mathrm{MeV} \right]$  energy range, whereas the latter has a good enough spatial resolution of $12 \, \mathrm{arcsec}$ (FWHM) in the $\left[ 15 \, \mathrm{keV}, 10 \, \mathrm{MeV} \right]$ energy range once it is on source (active follow-up).

% As GRBs will then be monitored thanks to the SPI-ACS and IBIS instruments embarked on the \textsc{INTEGRAL} satellite mission, we thus have to convert the energy information into some constraints acting on the detectable energy flux by those instruments. The emitted EM flux is hence truncated to the \textsc{INTEGRAL} finite energy band $\left[ 75 \, , 2000 \right] \, \mathrm{keV}$. \\


\section*{Results}

\subsection*{Prompt -- Passive follow-up}

\begin{center}\vspace{.5cm}
    \includegraphics[scale=.6]{figures/covered_region.png}\includegraphics[scale=.45]{figures/fluence_distribution_12_pe.png}\includegraphics[scale=.37]{figures/sensitivity_distribution_pe.png}
    \captionof{figure}{(\textit{Left}) Cumulative distribution of the fraction of the LIGO/Virgo localization region accessible to INTEGRAL pointed follow-up observation in the LIGO/Virgo O2 run. (\textit{Middle}) Cumulative distribution of the detected fluences by INTEGRAL in the LIGO/Virgo O2 run.(\textit{Right}) Cumulative distribution of the sensitivity of the IBIS and SPI-ACS instruments.}
    \label{covered_region}
\end{center}

Choosing an isotropic sky distribution of our GW events, we show that INTEGRAL
detects 55 \% of the events with a $20 \, \sigma$-significance for faint event
and 10 \% of the events with a $30 \, \sigma$-significance for loud events if
the aLIGO/Advanced Virgo network is considered.

We also show that it exists a sky region complementarity between the two on board
instruments. Indeed, one has an excellent FoV whereas the other has a better
resolution. That is why we achieve with a better sensitivity combining the two.

\subsection*{Afterglow -- Instrument repointing}

As stated before the X-ray emission we investigate occurs until few days or
weeks after the prompt emission. It thus enable us to re-point the INTEGRAL
satellite and to align the IBIS instrument onto the source. The observation
strategy is then conditioned by two factors. The first one is the error-box area
covered by INTEGRAL at the prompt emission time. We decide from the right figure
?? we can trigger a follow-up if more than 30 \% of the error box is initially
covered.

\begin{center}\vspace{.5cm}
    \includegraphics[width=20cm]{figures/significance_vs_rate_af.png}
    \captionof{figure}{toto.}
    \label{covered_region}
\end{center}

The left figure brings an answer to the second question : what is the duration
limit after the prompt emission within which we are able to detect the SGRB
source ?

% \section*{Conclusions}

% Among compact objects populations GRBs are the most promising astrophysical
% sources from which we can expect a gamma-ray emission loud enough to be
% associated to GW radiation sources and thus be detectable with adequate
% instruments. We show in this poster that INTEGRAL is the appropriate to ensure a
% follow-up for both the prompt and afterglow emissions, thanks to the IBIS and
% SPI-ACES instruments.


%----------------------------------------------------------------------------------------
%	ACKNOWLEDGEMENTS
%----------------------------------------------------------------------------------------

\vspace{10mm}
\noindent {\normalsize \textbf{Acknowledgements}}

{\footnotesize 
  This research was supported in part by the ASTERICS grant. The authors would like to thank Barbara Patricelli for sharing data and Philippe Laurent for fruitful discussions. }
  %We thank ASTERICS. Barbara Patricelli for sharing data, Philippe Laurent for discussions}

%----------------------------------------------------------------------------------------
%	REFERENCES
%----------------------------------------------------------------------------------------

%\nocite{*} % Print all references regardless of whether they were cited in the poster or not
\bibliographystyle{plain} % Plain referencing style
\bibliography{reference} % Use the example bibliography file sample.bib

%----------------------------------------------------------------------------------------

\end{multicols}
\end{document}
